\documentclass[10pt]{article}
\usepackage{fullpage, graphicx, url, textcomp, hyperref}
\setlength{\parskip}{1ex}
\setlength{\parindent}{0ex}
\begin{document}
\section*{GhostPDL}

This document discusses configuration, compilation and running of the
GhostPDL \footnote{PDL stands for Page Description Language} family of products:\emph {XPS}~\cite{XPSMAN}, \emph {PCL5E}~\cite{PCLTRM},
\emph{PCL5C}~\cite{PCL5CTRM}, \emph{PCLXL}~\cite{PCLXLTRM}, and
\emph{HPGL/2} with \emph{RTL}~\cite{RTLTRM}, additionally a scaled
down \emph{PJL}~\cite{PJLTRM} interpreter is provided.  The PDL
products use the Ghostscript Graphic Library for graphics, imaging and
driver support.  GhostPDL may be configured with PostScript and PDF support but these PDL's are not documented here, see the Ghostscript documentation at
\url{www.ghostscript.com/doc/} for information about these languages.
The relevant ghostscript version used by the PDL's  can be found
in gs/base/version.mak.  The GhostPDL product may be configured with 
PostScript and PDF support but that is not documented here.


GhostPDL is not intended to be a finished software application but a
collection of software components that will ultimately be included in
a software application or a real time embedded system. Generally the
GhostPDL languages are compliant with the Genoa (now QualityLogic)
Functional Test Suite and the Genoa Application Test suite.  Also each
language should run the corresponding Genoa Comprehensive Evaluation
Test (CET) without egregious errors, the following text files in the
pcl and pxl source code directories should be consulted for
discrepancy details: Anomalies.txt (PCL), pxcet.txt (PCLXL), and
pxfts.txt (PCLXL).

In this document and the software, PCL6 refers to technology that
supports both PCL5 and PCLXL languages.

\subsection*{Quick Start for Unix environment with GCC.}
\begin{verbatim}
# unpack the release and go to the release directory.
tar zxvf ghostpdl-xxx.tar.gz; cd ghostpdl-xxx
./configure
make pcl  # build pcl
make xps # build xps
\end{verbatim}

\subsection*{Supported development environments}
\begin{itemize}
\item Windows Vista, NT and '95 with Microsoft Visual C/C++
\item Linux with gcc
\item Unix environments without GCC
\end{itemize}

\subsection*{Unpacking the source}

The source code will either be on CD-ROM (without any archiving or
compression) or in a archived compressed format. Zip is used for
Windows environments and compressed tar format for Unix environments.
If you have a CD-ROM release you will want to replicate the directory
structure from the CD-ROM to a development directory on a hard drive.
The compile process does not require a special directory name.  If you
have obtained the archived compressed version use:

 For Windows:

 pkunzip -d $<$archive\_name$>$\\

 For Unix:: 

 gunzip $<$archive\_name.gz$>$\\ 
 tar xvf $<$archive\_name$>$ 
 
 After unpacking the result should be a directory hierarchy which is
 briefly described next.

\subsection*{Brief Overview of Directory Hierarchy}
\begin{description}

\item[xps] Source code for the XPS interpreter.  

\item[pcl] Source code for the PCL interpreter.  Files beginning with
  ``pc'' handle pcl state, text and rules.  ``rt'' files implement pcl's
  raster language RTL and the ``pg'' source code files are for HPGL/2.
  
\item[pxl] Source code for the PCL-XL interpreter.  This is an
  entirely different language than PCL; all files in this subdirectory
  are prefixed with ``px''
  
\item[pl] This subdirectory contains code that can be shared by PCL
  and PXL\@.  In particular font scaling code, the printer job language
  (PJL) and the language switching machinery are in this subdirectory.
  All files in this directory are prefixed with pl.
  
\item[common] PCL and PXL shared makefiles.  The subdirectory is an
  historical artifact from when PCL and PXL had separate makefiles.
  It does contain the makefile to build the ghostscript graphics
  library.

\item[tools] miscellaneous tools and test files.
  
\item[tools/GOT] GOT means graphical object tagging.  The system is
  able to classify high level graphical objects (text, images and
  vectors) and store this information in the framebuffer along with
  the output data.  The tools in these directories are used to unpack
  a framebuffer that has been tagged for display in a viewable format.
  
\item[tools/viewer] The Java pcl viewer is a sample graphical user
  interface for the GhostPCL system.
  
\item[language\_switch] top level makefiles for building a share
  language environment.  This is where the shared language environment
  which includes \emph{PostScript} and PDF can be built.
  
\item[gs] Ghostscript branch, refer to the documentation at
  \url{www.ghostscript.com/doc/} for more information about
  ghostscript.
  
\item[main] The ``workhorse'' makefiles of the system.  These are also
  most likely to need modification to add/remove devices, choose a
  font scaler, and set directory locations.

\item[urwfonts] 80 URW TrueType fonts.

\item[win32] Microsoft Visual C project files and instructions (ReadMe.txt) how to use them.

\end{description}

This directory structure is the default, but the directories can be
rearranged with minor modifications to the makefiles.

\subsection*{Building with Microsoft Visual C/C++}

The recommended way to build is to use the project files found in the subdirectory win32, there is a ReadMe.txt file containing instructions.  The rest of this section may prove helpful if you wish to wrestle with Microsoft's NMAKE.
  
The GhostPDL tools are known to build with Visual C/C++ 4.0, 5.0 and 6.0, Visual Studio 2005 and 2008. \footnote{A subset of the components have been known to build with these tools}.  Automatic makefile wrapping to projects file stopped working some time ago, we aren't sure which Visual Studio version deprecated the functionality.

There are 2 ways of building the PCL components with Microsoft Visual
C/C++: Convert the current makefile environment to a Visual C/C++
project; or compile the source directly using nmake. The following
steps take you through converting the supplied makefile environment to
a Visual C++ project:

\begin{itemize}
\item  Unpack the source code, see Unpacking the source.

\item  Start MSVC++.
  
\item Open main/pcl6\_msvc.mak and MSVC++ should wrap the makefile
  automatically and create an MSVC project.
  
\item Set the project setting (Alt F7).  To create a debug:
  
  \textbf{NMAKE /f pcl6\_msvc.mak DEBUG=1
    DEVSTUDIO=c:$\backslash$progra~1$\backslash$micros~3}\footnote{DEBUG=1/0
    on/off will require you manually clean; del
    main$\backslash$obj$\backslash$*.*}
  
  Now set the name of the executable for debuggging to
  $.$$\backslash$obj$\backslash$pcl6.exe, and set the program argument
  to be any pcl6 options wanted and the name of the pcl file.

\item Now PCL6 is set to be compiled, debugged and use other features
  of the MSVC++ IDE.\footnote{Dos hackers can start up a dos window,
    set the DEVSTUDIO environment variable, and use the same make
    commands as above (the DEVSTUDIO variable should be set to the top
    level of the Microsoft Developer Studio, e.g. set
    DEVSTUDIO=c:$\backslash$progra~1$\backslash$devstu~1'' or NMAKE /f
    pcl6\_msvc.mak DEBUG=1
    DEVSTUDIO=c:$\backslash$progra~1$\backslash$micros~3)}

\end{itemize}

This will build both the PCL and XL language and supporting language
switching code.

\subsection*{Building with Linux and GCC}

The pcl tools have been compiled on Linux using GCC. It is easiest to
simply use the instructions provided in the section Quick Start For
Unix environment with GCC.

\subsection*{Building on a generic UNIX platform}

The PCL tools have been compiled on Solaris with the Sun Development
Tools and SGI with the native SGI C compiler. We only provide
makefiles for the gcc tools and assume users can customize the gcc
makefiles such that they work with vendor's compilers. For the Sun
Development tools the following workaround is sufficient to build the
software:

\begin{itemize}
\item change to the main subdirectory.
\item make CC\_=cc CCLD=cc $\backslash$ CCAUX=cc CFLAGS=-g CC\_NO\_WARN=cc GCFLAGS= 
\end{itemize}

If you intend to do development using the software a dedicated
makefile should be constructed.

\subsection*{Customizing the build process}

The build process is completely configurable. Here is a list of
things that are user will most like be interested in customizing.  To
change any of these see the appropriate makefile for your platform in
the ``main'' subdirectory

\begin{itemize}
\item  Directory location of source directories.
\item  Directory location of objects, executables, and other compile time generated files.
\item  Selection of devices.
\item  Directory location of PJL filesystem volume 0 and 1
\item  The font scaling technology.
\begin{itemize}
\item  Defaults to /tmp/pjl0 and /tmp/pjl1
\item  Edit pl/pl.mak PJL\_VOLUME\_0 and PJL\_VOLUME\_1 to match desired root
\end{itemize}

\end{itemize}

All of these can be configured in the top level makefile or can be
specified on the make command line.

Sample make targets include: 
\begin{description}
\item[debug] build tools with symbols and debugging information
\item[product] builds optimized code.
\item[pdl-pg] builds profiling.
\end{description}

For a complete list of targets see the top level makefile named Makefile.  Each target is somewhat self explanatory.

\subsection*{Building only one Language}
PCL or PCL-XL can be built together in a language switching
environment or each can be built alone with the supporting PJL
interpreter.  The simplest way to implement one language is simply to
remove the unwanted implementation from the pdl implementation table
located in pl/plimpl.c:

For example, this is the default table with two implementations: PCL and XL. 

\begin{tabbing}
\hspace{.5in}\=\hspace{.5in}\=\\
\> /* Zero-terminated list of pointers to implementations */\\
\> pl\_interp\_implementation\_t const * const pdl\_implementation[] = \{ \\ 
\> \> \&pcl\_implementation, \\ 
\> \> \&pxl\_implementation, \\ 
\> \> 0 \\ 
\> \};
\end{tabbing}

If you only wish to use one interpreter remove the unwanted one and
recompile the code.

\subsection*{Running the products.}
\begin{quote}
Few things are harder to put up with than a good example.
\emph{Mark Twain}
\end{quote}


Most ghostscript options, as described in the Ghostscript
documentation \url{www.ghostscript.com/doc/}, have similar effect in
the GhostPCL system.  Of course, options specific to the PostScript or
PDF language are not relevant and are ignored.


\textbf{pcl6 mypcl.pcl}

Interpret a pcl file called mypcl.pcl and render it to the default
device, usually a simple display  is the default, X11 on Unix like systems and the Window's display device on Windows.

\textbf{pcl6 -dTextAlphaBits=4 mypcltext.pcl}

When rendering pcl text on a low (screen) resolution display device,
use the TextAlphaBits option to enable anti-aliasing.

\textbf{pcl6 -sDEVICE=ljet4 -sOutputFile=''$|$ lpr'' -dNOPAUSE mypcl.pcl}

Interpret mypcl.pcl and send the Laserjet 4 formatted output to the
command lpr.

\textbf{pcl6 -sDEVICE=pcxcmyk -sOutputFile=''pcxpage.\%d'' -dNOPAUSE mypcl.pcl}

Interpret mypcl.pcl and generate CMYK output. Pages are to be put in
files named pcxpage.1, pcxpage.2, pcxpage.3, etc.

\textbf{pcl6 -r72 -sDEVICE=x11mono mypcl.pcl -r100 -sDEVICE=x11 mypcl.pcl}

Render a pcl file at 72dpi on the monochrome X11 device, then render
the same file at 100 dpi on color X11 device. This demonstrates
on-the-fly device switching.

\textbf{./pcl6 -J"@PJL SET SYMSET = ISOL1" mypcl.pcl}

PJL or PCL Job control commands can be set directly on the command line.  This example sets the default symbol set to ISO Latin 1.

\textbf{pcl6 -sDEVICE=pdfwrite -sOutputFile=mypcl.pdf mypcl.pcl}

Convert the pcl file mypcl.pcl to PDF with output written to
mypcl.pdf.

\textbf{pcl6}

Simply running the interpreter should generate some useful information
about the available options and devices.  

\subsection*{PCL Personality}

The PCL emulation comes in three flavors: PCL5E, PCL5C, and RTL. The
PCL5E personality thresholds colors to black and white irrespective of
the color parameters of the output device. PCL5C is the color
personality, used with a monochrome device it will grayscale colors.
The RTL personality can be used to print HPGL/2 RTL plot files.

\textbf{pcl6 -lRTL myrtl.rtl}

run the interpreter with the rtl personality.

\textbf{pcl6 -lPCL5E -sDEVICE=ljet4 mypcl.pcl}

run the interpreter with the pcl5e personality. This will threshold
colors to black and white (ljet4 is a 1 bit device).

\textbf{pcl6 -lPCL5C -sDEVICE=ljet4 mypcl.pcl}

run the interpreter with the pcl5c personality. This will grayscale
colors on the 1 bit output device.  If not set on the command line the
pcl interpreter personality will be set to PCL5E if the output device
is 1 bit per pixel otherwise it is set to PCL5C. RTL must be
explicitly set on the command line. RTL always grayscales and never
thresholds colors to black and white.

\textbf{pcl6 -H12x12x12x12 mypcl.pcl}

apply hardware margins of 1/6 inch, the parameter's units are points.
Without this or specification of hardware margins from a Ghostscript
device the margins will be 0 or full bleed.  Most HP printers have a
hardware margin greater than 0.  Normally this would be expressed by
setting a device parameter on the command line but PCL does not parse
this type (array) parameter yet.

\subsection*{Fonts}

The release is packaged with 80 high quality URW TrueType fonts. For
commercial use of the GhostPCL technology these fonts can be licensed
from Artifex. The fonts are searched for in either the fonts,
/windows/fonts, or a directory specified with the PCLFONTSOURCE
environment variable.  For historical reasons the directory path must
be specified using forward slashes and must include a trailing slash.

Fonts and a font scaler from a third-party vendor such as Agfa or
Bitstream may also be used. There is an existing interface for
integrating the AGFA Universal Font Scaler Technology, several Artifex
customers currently use this solution. The software can use Hewlett
Packard FONTSMART version 1.5 or Windows TrueType fonts, using either
of these font solutions require minor PCL code modifications.

\subsection*{PCL Code changes required to use other TrueType fonts.}

To use a new set of TrueType fonts requires modifying the C code in
the file pl/plftable.c. The C structure resident\_table contains a
list of Windows TrueType font names. In the released package these
names will correspond with the Windows True Type font names in the URW
font set. To use a different font set these names must be replaced
with the new font names and the code (at least the plftable.c module)
should be recompiled and linked.  The file tools/fontpage.pcl can be
run to display font samples and the pcl escape sequences required to
select the fonts in a pcl stream.  This option is only recommended for
advanced developers.  It is easiest to simply create a downloaded font
and embed it in the PCL stream, see the PCL Technical Reference
Manual~\cite{PCLTRM} for details.

\subsection*{Using the language switching build}

For printer and embedded device users we provide a complete language
switching solution consisting of PCL/HPGL2, PCLXL, PS, PDF, with Job
Control.  For host based user we strongly recommend that you use
GhostPCL and Ghostscript separately as the shared language has
particular feature well suited to printer environment but the same
feature may produce unexpected results on host based environments.

\subsection*{Building the Language Switch Environment}

At this time, we have makefile support for Microsoft Visual C and
Linux with gcc.  For the Microsoft Visual C the use the instructions
in the section Building with Microsoft Visual C, but this time you
will wrap the makefile pspcl6\_gcc.mak which is located in the
directory language\_switch.  For Linux the makefile targets for the
language switch build are exactly the same except each target is
prefixed with ``ls\_''.  So using the pattern from the ``Quick Start''
section we have:

\begin{verbatim}
tar zxvf ghostpcl-xxx.tar.gz; cd ghostpcl-xxx  # unpack the release and go to the release directory.
make ls_fonts # install the fonts.
make ls_product #compile and link pspcl6.
make ls_test # test pspcl6 (optional).
make ls_install # install it.
\end{verbatim}

\subsection*{Reporting bugs}

If you find a bug or have comments about this documentation, please
send mail to bug-pcl@ghostscript.com.

\subsection*{Trademark Credits}

\emph{PostScript} is a registered trademark of Adobe Systems Inc.
\emph{PCL} is a registered trademark of Hewlett-Packard Company.

\begin{thebibliography}{99}
  
\bibitem{XPSMAN}\emph{XML Paper Specification v.1.0}
  
\bibitem{PCLTRM} \emph{PCL 5 Printer Language Technical Reference
    Manual}, HP Part No. 5961-0509, First Edition - October 1992
  
\bibitem{PCL5CTRM} \emph{PCL 5 Color Technical Reference Manual}
  Copyright 1999, Hewlett-Packard Company.

\bibitem{PCLXLTRM} \emph{PCL XL Feature Reference}.
  
\bibitem{PJLTRM} \emph{Printer Job Language Technical Reference
    Manual} Edition 10, HP Part No. 5021-0380, October 1997.
  
\bibitem{RTLTRM} \emph{The HP-GL/2 and HP RTL Reference Guide A
    Handbook for Program Developers}, Addison Wesley Publishing 
  Company, 1993.

\url{http://www.microsoft.com/whdc/xps/xpsspec.mspx}

\end{thebibliography}

\end{document}
